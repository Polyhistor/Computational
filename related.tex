
A considerable amount of research has focused on enhancing deep learning's computational efficiency and scalability, particularly through advancements in numerical methods and distributed computing. 
The work by~\cite{najafabadi2015deep} provides an extensive overview of deep learning applications in big data analytics, emphasizing the inherent challenges in managing large-scale data 
and the computational power required. The authors discuss various deep learning architectures and the specific numerical methods used to optimize these models, setting a foundation for understanding 
the computational needs of big data-driven deep learning.

The book by~\cite{yan2023computational} presents a comprehensive discussion on the computational methods for deep learning, detailing the theoretical aspects of optimization algorithms and their implementation in practical scenarios. This work bridges the gap between theory and practical deployment, offering insights into the challenges of implementing these methods in a distributed environment. The book highlights the importance of selecting appropriate numerical methods to ensure both convergence and computational efficiency.

A survey by~\cite{zhang2023distributed} delves into distributed deep learning frameworks, discussing the evolution from traditional distributed machine learning to more sophisticated distributed deep learning systems. It explores various distributed computing techniques such as federated learning, GPU acceleration, and parallel processing, which are essential for scaling deep learning models for big data applications. The survey compares different distributed frameworks, analyzing their scalability, efficiency, and suitability for diverse deep learning tasks.

Similarly,~\cite{li2019federated} provides a foundational overview of federated learning, a decentralized approach to training models without sharing raw data between nodes. This technique is especially useful for privacy-sensitive applications in big data. The authors discuss federated learning’s architecture, key challenges, and promising results in scaling deep learning for real-world applications.

\cite{li2020survey} offers a detailed survey of scalable deep learning techniques, specifically focusing on efficient parallel processing and distributed systems. The work discusses both hardware-based approaches, such as GPU acceleration, and software-based frameworks like Apache Spark, which have shown promise in reducing the computational time required for large-scale models, making deep learning more feasible for real-time applications.

Further,~\cite{ben2019demystifying} provides insights into optimization methods specifically tailored for big data in deep learning. The authors review key numerical methods and optimization algorithms, addressing their impact on model convergence and performance. This paper is particularly valuable for understanding the trade-offs between computational cost and accuracy, which are central to deep learning in big data contexts.

Overall, the related work in this domain underscores the interplay between numerical optimization techniques and distributed computing as fundamental enablers of scalable deep learning. These works collectively highlight the importance of computational efficiency, scalability, and the need for continued research to address the complexities of big data-driven deep learning.


